% Options for packages loaded elsewhere
\PassOptionsToPackage{unicode}{hyperref}
\PassOptionsToPackage{hyphens}{url}
%
\documentclass[
]{article}
\usepackage{amsmath,amssymb}
\usepackage{lmodern}
\usepackage{iftex}
\ifPDFTeX
  \usepackage[T1]{fontenc}
  \usepackage[utf8]{inputenc}
  \usepackage{textcomp} % provide euro and other symbols
\else % if luatex or xetex
  \usepackage{unicode-math}
  \defaultfontfeatures{Scale=MatchLowercase}
  \defaultfontfeatures[\rmfamily]{Ligatures=TeX,Scale=1}
\fi
% Use upquote if available, for straight quotes in verbatim environments
\IfFileExists{upquote.sty}{\usepackage{upquote}}{}
\IfFileExists{microtype.sty}{% use microtype if available
  \usepackage[]{microtype}
  \UseMicrotypeSet[protrusion]{basicmath} % disable protrusion for tt fonts
}{}
\makeatletter
\@ifundefined{KOMAClassName}{% if non-KOMA class
  \IfFileExists{parskip.sty}{%
    \usepackage{parskip}
  }{% else
    \setlength{\parindent}{0pt}
    \setlength{\parskip}{6pt plus 2pt minus 1pt}}
}{% if KOMA class
  \KOMAoptions{parskip=half}}
\makeatother
\usepackage{xcolor}
\usepackage[margin=1in]{geometry}
\usepackage{longtable,booktabs,array}
\usepackage{calc} % for calculating minipage widths
% Correct order of tables after \paragraph or \subparagraph
\usepackage{etoolbox}
\makeatletter
\patchcmd\longtable{\par}{\if@noskipsec\mbox{}\fi\par}{}{}
\makeatother
% Allow footnotes in longtable head/foot
\IfFileExists{footnotehyper.sty}{\usepackage{footnotehyper}}{\usepackage{footnote}}
\makesavenoteenv{longtable}
\usepackage{graphicx}
\makeatletter
\def\maxwidth{\ifdim\Gin@nat@width>\linewidth\linewidth\else\Gin@nat@width\fi}
\def\maxheight{\ifdim\Gin@nat@height>\textheight\textheight\else\Gin@nat@height\fi}
\makeatother
% Scale images if necessary, so that they will not overflow the page
% margins by default, and it is still possible to overwrite the defaults
% using explicit options in \includegraphics[width, height, ...]{}
\setkeys{Gin}{width=\maxwidth,height=\maxheight,keepaspectratio}
% Set default figure placement to htbp
\makeatletter
\def\fps@figure{htbp}
\makeatother
\setlength{\emergencystretch}{3em} % prevent overfull lines
\providecommand{\tightlist}{%
  \setlength{\itemsep}{0pt}\setlength{\parskip}{0pt}}
\setcounter{secnumdepth}{-\maxdimen} % remove section numbering
\ifLuaTeX
  \usepackage{selnolig}  % disable illegal ligatures
\fi
\IfFileExists{bookmark.sty}{\usepackage{bookmark}}{\usepackage{hyperref}}
\IfFileExists{xurl.sty}{\usepackage{xurl}}{} % add URL line breaks if available
\urlstyle{same} % disable monospaced font for URLs
\hypersetup{
  pdftitle={Global Crisis Data Bank - Earthquake Impacts},
  pdfauthor={Hamish Patten and Justin Ginnetti},
  hidelinks,
  pdfcreator={LaTeX via pandoc}}

\title{Global Crisis Data Bank - Earthquake Impacts}
\author{Hamish Patten and Justin Ginnetti}
\date{2023-08-21}

\begin{document}
\maketitle

{
\setcounter{tocdepth}{2}
\tableofcontents
}
\hypertarget{initial-results}{%
\section{Initial Results}\label{initial-results}}

In this article, we include some preliminary plots to explore the impact data currently contained in the Global Crisis Data Bank. Considerable future modifications will be made, which will almost certainly result in different final results than what is shown here.

\hypertarget{impact-data}{%
\subsection{Impact Data}\label{impact-data}}

Figure \ref{fig:noConts} indicates the number of impact records per continent. Note that the continent definitions applied here split the North and South Americas, splitting the Central Americas into one of the two.

\begin{figure}
\includegraphics{EQ_EDA_files/figure-latex/noConts-1} \caption[Number of impact records per continent, for earthquakes in the GCDB database]{Number of impact records per continent, for earthquakes in the GCDB database.}\label{fig:noConts}
\end{figure}

We can also take the individual country ISO3C codes and make a wordcloud with the number of events per country (log-scale), shown in figure \ref{fig:wordclISO}. The results show that countries in the Central Americas tend to dominate the database, mostly due to the presence of the Desinventar database which was first established in the Central Americas.

\begin{figure}
\includegraphics{EQ_EDA_files/figure-latex/wordclISO-1} \caption[A wordcloud of the number of impact records per country (represented via the ISO3C code), for earthquakes in the GCDB database]{A wordcloud of the number of impact records per country (represented via the ISO3C code), for earthquakes in the GCDB database.}\label{fig:wordclISO}
\end{figure}

The number of impact records per year since 1900, on a log-scale, is shown in figure \ref{fig:noYears}. The number of earthquake impact records has increased exponentially since 1900. Such an exponential relationship is reflective of the start of journey of consistent and complete impact recording on a global level. In moving towards an ideal future, with a perfected impact recording process in place on a global level, we would see this exponential relationship tailing off to a stable value (to converge to an extreme value distribution).

\begin{figure}
\includegraphics{EQ_EDA_files/figure-latex/noYears-1} \caption[Number of impact records per year, for earthquakes in the GCDB database]{Number of impact records per year, for earthquakes in the GCDB database.}\label{fig:noYears}
\end{figure}

Each of the impact estimate databases that are currently included in the GCDB have different levels of spatial decomposition of each impact estimates. Figure \ref{fig:noADM} reflects the number of impacts recorded per database, per spatial level disaggregation, for earthquakes in the GCDB. The majority of the impact records present in the GCDB come from the Desinventar database (UNDRR), the impact estimates from which are calculated and curated by local government organisations. More than half of the entries are disaggregated by admin level 2 for the Desinventar database, falling to roughly 40\% for EM-DAT, and the GIDD database is shown to always aggregate to national level.

\begin{figure}
\includegraphics{EQ_EDA_files/figure-latex/noADM-1} \caption[Number of impact records per impact database, per spatial resolution (administrative level) for earthquakes in the GCDB database]{Number of impact records per impact database, per spatial resolution (administrative level) for earthquakes in the GCDB database.}\label{fig:noADM}
\end{figure}

We can also break down the impact recordings according to the GCDB impact taxonomy. The impact category is the top of the classification hierarchy (the trunk of the tree in a branching process), with four categories. Figure \ref{fig:impcats} shows a breakdown of the number of recorded impacts based on the impact category, reflecting that the majority of the estimates are related to the impacted population. This figure reflects that environemental assets are not often recorded by any of the databases currently present in the GCDB.

\begin{figure}
\includegraphics{EQ_EDA_files/figure-latex/impcats-1} \caption[Number of impact records per impact category, for earthquakes in the GCDB database]{Number of impact records per impact category, for earthquakes in the GCDB database.}\label{fig:impcats}
\end{figure}

Breaking down the impact estimate classification into further sub-categories, then combining with the impact type, shown in figure \ref{fig:impactsunmat}. This plot shows that, although estimating impacts on the population has been a key priority of the databases brought into the GCDB, the most prominent single impact was actually the aggregated number of buildings damaged.

\begin{figure}
\includegraphics{EQ_EDA_files/figure-latex/impactsunmat-1} \caption[Number of unmatched impact records per impact, for earthquakes in the GCDB database]{Number of unmatched impact records per impact, for earthquakes in the GCDB database.}\label{fig:impactsunmat}
\end{figure}

\hypertarget{matched-hazard-impact-data}{%
\subsection{Matched Hazard-Impact Data}\label{matched-hazard-impact-data}}

By attempting to match each impact record with an associated hazard footprint from the USGS shakemap catalog, we find that a significant number (18.0594\%) of impact recordings remain unmatched. Therefore, this has an impact on some of the summary statistics that we displayed above. For example, we can make the exact same plot as figure @ref(fig:impacts\_unmat) but for only the impact data that was matched to a hazard footprint. Figure \ref{fig:impactsmat} shows the number of recorded events that have been matched with an associated hazard footprint. The number of earthquake events with matched hazard-impact data now reflects that deaths is the most preminent impact in the GCDB.

\begin{figure}
\includegraphics{EQ_EDA_files/figure-latex/impactsmat-1} \caption[Number of matched impact records per impact, for earthquakes in the GCDB database]{Number of matched impact records per impact, for earthquakes in the GCDB database.}\label{fig:impactsmat}
\end{figure}

We can also visualise the matched earthquake events with respect to their alertscore as provided by USGS PAGER, in figure \ref{fig:pager}. As may be expected, the number of events in the GCDB decreases as the alertscore increases. Note that the alertscore severity may not actually correlated with the observed impact value, but just reflects that the parameterisation ensures that only a few events make it into the more severe categories such as red.

\begin{figure}
\includegraphics{EQ_EDA_files/figure-latex/pager-1} \caption[Number of impacts matched with earthquake shakemaps per PAGER alertscore, for earthquakes in the GCDB database]{Number of impacts matched with earthquake shakemaps per PAGER alertscore, for earthquakes in the GCDB database.}\label{fig:pager}
\end{figure}

To visualise what the correlation is between the PAGER alertscore and some of the observed impacts, we use box-and-whisker diagrams, see figure \ref{fig:pagerbox}. The alertscore seems to correlate well with the number of deaths, which is what the categorisation was built on, but for the other impacts the performance is relatively poor in separating impact severity.

\begin{figure}
\includegraphics{EQ_EDA_files/figure-latex/pagerbox-1} \caption[Box and whisker diagrams of the impact values against the assigned PAGER alertscore, for matched impact-hazard earthquake shakemaps in the GCDB database]{Box and whisker diagrams of the impact values against the assigned PAGER alertscore, for matched impact-hazard earthquake shakemaps in the GCDB database.}\label{fig:pagerbox}
\end{figure}

Not only do we have access to the PAGER alertscore, but also additional hazard-specific information such as the maximum hazard intensity (MMI). A plot of the number of events per maximum hazard intensity is shown in figure \ref{fig:matchint}, reflecting that the majority of the events in the database (the median) have a maximum earthquake intensity of around 7 MMI. Note that historical impacts before somewhere around 2000-2010 have the hazard intensity rounded to half-integer values.

\begin{figure}
\includegraphics{EQ_EDA_files/figure-latex/matchint-1} \caption[Number of impacts matched with earthquake shakemaps per earthquake intensity, for earthquakes in the GCDB database]{Number of impacts matched with earthquake shakemaps per earthquake intensity, for earthquakes in the GCDB database.}\label{fig:matchint}
\end{figure}

We can also compare the earthquake magnitude (on the Richter scale), to the earthquake intensity (MMI), see figure \ref{fig:intmag}. There is a strong correlation between the two values (adjusted R² value = 0.98 and p-value \textless{} 2 x 10\^{}(-16)).

\begin{figure}
\includegraphics{EQ_EDA_files/figure-latex/intmag-1} \caption[Earthquake intensity against earthquake magnitude for the matched hazards]{Earthquake intensity against earthquake magnitude for the matched hazards}\label{fig:intmag}
\end{figure}

In terms of the matched hazard-impact data, we can see that events that occurred in Asia seem to have a more successful matching rate.

\begin{figure}
\includegraphics{EQ_EDA_files/figure-latex/matchcont-1} \caption[Number of impacts matched with earthquake shakemaps per continent, for earthquakes in the GCDB database]{Number of impacts matched with earthquake shakemaps per continent, for earthquakes in the GCDB database.}\label{fig:matchcont}
\end{figure}

For many of the matched hazard impact events, we have associated secondary or triggering hazards. Figure \ref{fig:haz_link} shows the hazard code from the HIPS (UNDRR-ISC) taxonomy, and also disaggregates by whether the USGS hazard data announced a tsunami alert.

\begin{figure}
\includegraphics{EQ_EDA_files/figure-latex/haz_link-1} \caption[Frequency table of the number of linked secondary hazards, from both the impact and hazard data]{Frequency table of the number of linked secondary hazards, from both the impact and hazard data.}\label{fig:haz_link}
\end{figure}

We are also interested in seeing whether the maximum earthquake hazard intensity correlates with the impact magnitude. Figure \ref{fig:intdeat} shows the poor correlation between the number of deaths and the maximum hazard intensity for the matched earthquake events. There is no statistically significant evidence supporting a non-zero correlation (adjusted R² value = -0.0015994 and p-value 0.43), reflecting the need to include more information, such as the exposed population or proxies for coping-capacity, before stronger correlation is expected.

\begin{figure}
\includegraphics{EQ_EDA_files/figure-latex/intdeat-1} \caption[Number of deaths against earthquake intensity (MMI) for the matched hazards in the GCDB]{Number of deaths against earthquake intensity (MMI) for the matched hazards in the GCDB}\label{fig:intdeat}
\end{figure}

However, to show one example where a correlation may be visible, figure \ref{fig:intaid} shows the correlation between maximum hazard intensity and the cost of international aid allocated to the event.

\begin{figure}
\includegraphics{EQ_EDA_files/figure-latex/intaid-1} \caption[Disaster aid allocated against earthquake intensity (MMI) for the matched hazards in the GCDB]{Disaster aid allocated against earthquake intensity (MMI) for the matched hazards in the GCDB}\label{fig:intaid}
\end{figure}

\hypertarget{geospatial-analysis}{%
\subsection{Geospatial Analysis}\label{geospatial-analysis}}

So how about the spatial disaggregation of the number of earthquake events with impact data in the GCDB? Figure \ref{fig:mapnoevs} shows the number of events with impact records per country, aggregated over all time.

\begin{figure}
\includegraphics{EQ_EDA_files/figure-latex/mapnoevs-1} \caption[Map of the number of impact records per country, for earthquakes]{Map of the number of impact records per country, for earthquakes}\label{fig:mapnoevs}
\end{figure}

Not only is it important to try to visualise the number of events with impact records, but the number of events that have both impact and hazard data matched, shown in figure \ref{fig:mapnoevsmat}.

\begin{figure}
\includegraphics{EQ_EDA_files/figure-latex/mapnoevsmat-1} \caption[Map of the number of hazard-matched impact records per country, for earthquakes]{Map of the number of hazard-matched impact records per country, for earthquakes}\label{fig:mapnoevsmat}
\end{figure}

So how about plotting the total number of deaths?

\begin{figure}
\includegraphics{EQ_EDA_files/figure-latex/mapdeaths-1} \caption[Map of the total number of recorded deaths per country, for earthquakes]{Map of the total number of recorded deaths per country, for earthquakes}\label{fig:mapdeaths}
\end{figure}

\hypertarget{ifrc-appeal-field-report-data---earthquake-only}{%
\subsection{IFRC Appeal \& Field Report Data - Earthquake Only}\label{ifrc-appeal-field-report-data---earthquake-only}}

\begin{figure}
\includegraphics{EQ_EDA_files/figure-latex/ifrccont-1} \caption[Number of impact records per continent, for earthquakes in the IFRC Appeals and Field Reports database]{Number of impact records per continent, for earthquakes in the IFRC Appeals and Field Reports database.}\label{fig:ifrccont}
\end{figure}

\begin{figure}
\includegraphics{EQ_EDA_files/figure-latex/ifrcappimp-1} \caption[Number of impact records per impact, for earthquakes in the IFRC Appeals and Field Reports database]{Number of impact records per impact, for earthquakes in the IFRC Appeals and Field Reports database.}\label{fig:ifrcappimp}
\end{figure}

\hypertarget{ifrc-field-reports-only-data---earthquakes}{%
\subsection{IFRC Field Reports Only Data - Earthquakes}\label{ifrc-field-reports-only-data---earthquakes}}

\begin{figure}
\includegraphics{EQ_EDA_files/figure-latex/ifrcfr-1} \caption[Number of field report entries per event, for earthquakes, disaggregated by impact type]{Number of field report entries per event, for earthquakes, disaggregated by impact type.}\label{fig:ifrcfr}
\end{figure}

\includegraphics{EQ_EDA_files/figure-latex/unnamed-chunk-1-1}

Per database, what percentage of the impact data had a matching hazard?

\end{document}
